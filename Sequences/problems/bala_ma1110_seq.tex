%\documentclass{article}
\documentclass[journal,12pt,onecolumn]{IEEEtran}
\usepackage{amsmath}
\usepackage{multicol}
\usepackage{enumerate}
\usepackage{iithtlc}

\begin{document}
\providecommand{\nCr}[2]{\,^{#1}C_{#2}} % nCr
\providecommand{\nPr}[2]{\,^{#1}P_{#2}} % nPr
\providecommand{\mbf}{\mathbf}
\providecommand{\pr}[1]{\ensuremath{\Pr\left(#1\right)}}
\providecommand{\qfunc}[1]{\ensuremath{Q\left(#1\right)}}
\providecommand{\sbrak}[1]{\ensuremath{{}\left[#1\right]}}
\providecommand{\lsbrak}[1]{\ensuremath{{}\left[#1\right.}}
\providecommand{\rsbrak}[1]{\ensuremath{{}\left.#1\right]}}
\providecommand{\brak}[1]{\ensuremath{\left(#1\right)}}
\providecommand{\lbrak}[1]{\ensuremath{\left(#1\right.}}
\providecommand{\rbrak}[1]{\ensuremath{\left.#1\right)}}
\providecommand{\cbrak}[1]{\ensuremath{\left\{#1\right\}}}
\providecommand{\lcbrak}[1]{\ensuremath{\left\{#1\right.}}
\providecommand{\rcbrak}[1]{\ensuremath{\left.#1\right\}}}
\newcommand{\sgn}{\mathop{\mathrm{sgn}}}
\providecommand{\abs}[1]{\left\vert#1\right\vert}
\providecommand{\res}[1]{\Res\displaylimits_{#1}} 
\providecommand{\norm}[1]{\lVert#1\rVert}
\providecommand{\mtx}[1]{\mathbf{#1}}
\providecommand{\mean}[1]{E\left[ #1 \right]}
\providecommand{\fourier}{\overset{\mathcal{F}}{ \rightleftharpoons}}
%\providecommand{\hilbert}{\overset{\mathcal{H}}{ \rightleftharpoons}}
\providecommand{\system}{\overset{\mathcal{H}}{ \longleftrightarrow}}

\newcommand{\solution}{\noindent \textbf{Solution: }}
\providecommand{\dec}[2]{\ensuremath{\overset{#1}{\underset{#2}{\gtrless}}}}
\title{ 
\logo{
Problem Set: Sequences
}
}
\author{J.~Balasubramaniam$^{\dagger}$ %<-this  stops a space
\thanks{$\dagger$ The author is with the Department of Mathematics, IIT Hyderabad
502285 India e-mail: jbala@iith.ac.in. }
}
\maketitle
\begin{enumerate}
\setlength\itemsep{1.5em}
\item Show that the following sequences converge by the $\epsilon-K$ definition.
\begin{multicols}{2}
\begin{enumerate}[(i)]
\setlength\itemsep{2em}

\item
$
\lim_{n \rightarrow \infty} \frac{2n}{n+4\sqrt{n}} = 2
$
\item
$
\lim_{n \rightarrow \infty} \frac{10^7}{n} = 0
$ 
\item
$
\lim_{n \rightarrow \infty} \frac{n^2-1}{2n^2+3} = \frac{1}{2}
$
\item
$
\lim_{n \rightarrow \infty} \frac{3n+1}{2n+3} = \frac{3}{2}
$
\end{enumerate}
\end{multicols}
\item Show that the sequence $x_n = \frac{1}{\ln \brak{n+1}}$ converges to 0 using the $\epsilon-K$ definition.  Also find
the constant $K\brak{\epsilon}$ when $\epsilon = \frac{1}{2}$  and $\epsilon = \frac{1}{10}$.
\item Discuss the convergence/divergence of the following sequences $\brak{0 < a < 1}$ and $\brak{b > 1}$.
\begin{multicols}{3}
\begin{enumerate}[(i)]
\setlength\itemsep{2em}
\item
$
x_n = \frac{n^2}{n+5}
$
\item
$
x_n = \frac{n}{10^7}
$
\item
$
x_n = \sqrt{n+1}-\sqrt{n}
$
\item
$
x_n = \frac{(-1)^n}{n+1}
$
\item
$
x_n = \frac{1-2n}{1+2n}
$
\item
$
x_n = \frac{1-5n^4}{n^4+8n^3}
$
\item
$
x_n = \frac{\cos n}{n}
$
\item
$
x_n = \frac{1}{3^n}
$
\item
$
x_n = \frac{n^2}{e^n}
$
\item
$
x_n = a^n
$
\item
$
x_n = \frac{n!}{n^n}
$
\item
$
x_n = \frac{2^{3n}}{3^{2n}}
$
\item
$
x_n = \frac{n}{b^n}
$
\item
$
x_n = \frac{b^n}{n^2}
$
\item
$
x_n = \frac{5^n}{n!}
$

\end{enumerate}
\end{multicols}
\item Show that sequence is monotone and bounded.  Then find the limit.
\begin{multicols}{2}
\begin{enumerate}[(i)]
\item 
$
x_1 = 1;  x_{n+1} = \frac{x_n+1}{3}
$
\item$
x_1 = 2;  x_{n+1} = \sqrt{2x_n+1}
$
\end{enumerate}
\end{multicols}
\item Discuss whether the following sequences are Cauchy or not.	
\begin{multicols}{2}
\begin{enumerate}[(i)]
\setlength\itemsep{2em}
\item 
$
x_n = \frac{1}{n^2}
$
\item
$
x_n = 1 + \frac{1}{2!} + \frac{1}{3!} + \dots + \frac{1}{n!}
$
\item
$
x_n = \ln n^2 
$
\item
$
x_n = \sqrt{n}
$
\end{enumerate}
\end{multicols}
\item Show that  the sequence $x_n = \frac{4-7n^6}{n^6+3}$ converges using the $\epsilon-K$ definition.
\item Comment on the convergence of the sequence $x_n = \sin n$.
\item Does the recursively defined sequence $s_1 = 1; s_n = \frac{s_n+1}{5}$  converge?  If so, find its limit.
\end{enumerate}
\end{document}